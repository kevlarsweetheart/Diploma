%Nicht anfassen, so ist der Dokumentenaufbau
\documentclass[a4paper, 12pt]{article}
\usepackage[utf8]{inputenc} % Kodierung
\usepackage[T2A]{fontenc} % Explizite Nennung des Fonts
\usepackage[russian]{babel} % Sprache
\usepackage{graphicx} % immer benötigt für das Einbinden von Graphiken
\usepackage{blindtext} % Wenn man das Layout prüfen will, kann hier mit \blindtext Text eingfügt werden.
\usepackage{parskip} % Für den Abstand zwischen 2 Absätzen.
\setlength{\parskip}{12pt plus80pt minus10pt} % Genaue Einstellung von parskip
\usepackage{easy-todo} % Mit \todo{} Todos einfügen
\usepackage{csquotes} % Für ordentlichen Anführungszeichen
%\usepackage[iso, german]{isodate} % Für eine deutsche Formatierung des Abgabedatums / Eidesstattlicher Erkärung
%\usepackage[style=apa, backend=biber, sortlocale=ru_RU]{biblatex} % Biber backend für Literaturverzeichnis
%\addbibresource{literatur/bibliography.bib} % Einbinden der Literatur.
%\DeclareLanguageMapping{russian}{english} % Anpassen Spracheinstellungen im Literaturverzeichnis.
\usepackage[activate={true,nocompatibility},
	final,
	tracking=true,
	kerning=true,
	expansion=true,
	spacing=true,
	factor=1050,
	stretch=25,
	shrink=10]{microtype} % Für die Feineinstellung der Zeichensetzung.
\usepackage{booktabs}
\usepackage{appendix}
\usepackage[rflt]{floatflt}
\usepackage{fancyvrb}
\usepackage[hidelinks]{hyperref} % Klickbare aber nicht markierte Links im PDF
\usepackage{setspace}
\usepackage{fancyhdr} % Für schönere Kopf-/Fußzeilen und Fußnoten.
\usepackage[right=4 cm, left=2.5 cm, top=2.5 cm, bottom=3 cm]{geometry} % Seitenränder
\usepackage{pbox}
\usepackage{tabulary}
\sloppy
\fancyhf{}
\rfoot{\thepage}
\renewcommand{\headrulewidth}{0pt}
%Special cells with linebreaks possible
\newcommand{\specialcell}[2][c]{%
	\begin{tabular}[#1]{@{}t@{}}#2\end{tabular}}
%define blockquote-quotation environment
\renewenvironment{quotation}{
	\leftskip1cm
	\rightskip1cm
	\noindent
	\setstretch{1}
	\small
}

% define Footnote
\renewcommand\footnoterule{\kern-3pt \hrule width 3in height 0.7pt \hskip3pt \kern 2.6pt}
\let\oldfootnote\footnote
\renewcommand\footnote[1]{%
	\oldfootnote{\hspace{2mm}#1}}
%%%%%

% microtyping around some characters

%Extra-spacing around dash, and quotation-marks, and parentheses
\SetExtraKerning[unit=space]
	{
		encoding={*}, family={qhv}, series={b}, size={normalsize,large,Large}
	}
	{
		\textendash={400,400}, % for double-dash
		\textquotedblleft={ ,120}, % for left quotation-mark
		\textquotedblright={170, }, % for right quotation-mark
		"28={ ,250}, % left bracket, add space from right
		"29={300, } % right bracket, add space from left
	}

%%%%%


\makeatletter
\newcommand{\MSonehalfspacing}{%
  \setstretch{1.44}%  default
  \ifcase \@ptsize \relax % 10pt
    \setstretch {1.448}%
  \or % 11pt
    \setstretch {1.399}%
  \or % 12pt
    \setstretch {1.433}%
  \fi
}
\newcommand{\MSdoublespacing}{%
  \setstretch {1.92}%  default
  \ifcase \@ptsize \relax % 10pt
    \setstretch {1.936}%
  \or % 11pt
    \setstretch {1.866}%
  \or % 12pt
    \setstretch {1.902}%
  \fi
}
\newcommand{\MSverbatimspacing}{%
  \setstretch{1.44}%  default
  \ifcase \@ptsize \relax % 10pt
    \setstretch {1}%
  \or % 11pt
    \setstretch {1}%
  \or % 12pt
    \setstretch {1}%
  \fi
}
\makeatother

\begin{document}
\newgeometry{margin=2.5cm}
\begin{titlepage}
\thispagestyle{empty}
\newcommand{\HRule}{\rule{\linewidth}{0.5mm}}
\hspace{1cm}
\center

\textsc{\large Московский физико-технический институт \linebreak (государственный университет)}\\[1.5cm]
\textsc{ Факультет управления и прикладной математики}\\[0.8cm]
\MSonehalfspacing
\textsc{ Кафедра системного программирования}\\[1.0cm]

\HRule\\[1.4cm]
\MSdoublespacing
{\large Ефремова Мария Александровна}\\[0.5cm]
{ \huge \bfseries Использование распределения подграфов в графе для определения демографических атрибутов пользователей сети Интернет}\\[0.2cm]
{\large Выпускная квалификационная работа бакалавра}

\HRule \\[2.4cm]
\MSonehalfspacing

\begin{minipage}[t]{0.5\textwidth}
	\begin{itemize}
	\item[\emph{\bfseries Научные руководители:}] к.ф.-м.н. Турдаков Денис Юрьевич
	\item[\emph{}] Дробышевский Михаил Дмитриевич
	\end{itemize}
\end{minipage}

\vspace{2.9cm}

\flushright \emph{} Москва 2018
\end{titlepage}
\restoregeometry

\setstretch{3}
\microtypesetup{protrusion=false}
\tableofcontents
\microtypesetup{protrusion=true}
\thispagestyle{empty}

\MSonehalfspacing
\newpage
\setcounter{page}{1}
\pagestyle{fancy}
\setcounter{page}{1}
%%%%%%%%%%%%%%%%%%%%%%%%%%%%%
%%%Ab hier Inhalt einfügen%%%
%%%%%%%%%%%%%%%%%%%%%%%%%%%%%

\section{Введение} 

Пользователи сети Интернет принимают активное участие в создании контента, например: оставляют комментарии в социальных сетях, пишут отзывы на товары интернет-магазинов, ведут блоги и общаются на форумах. Однако многие ресурсы дают возможность не только делиться текстовой информацией, но и оставлять персональные данные - заполнять профиль. Как правило, к таким данным относятся имя, возраст, пол, контактная информация, интересы и прочее.

Информация в профиле, может оказаться неполной, например некоторые поля могут быть необязательными для заполнения, часто их оставляют пустыми. Более того, некоторые пользователи намеренно оставляют неверные данные. Возникает задача автоматического предсказывания недостающих демографических атрибутов, так как, к примеру, для проведения социологических исследований, а также для работы гибридных и основанных на знаниях рекомендательных систем и таргетированной рекламы необходим наиболее полный набор характеристик [1, 2].  

Во многих работах используются методы машинного обучения для решения этой задачи. Сначала осуществляется сбор данных для построения модели, после чего производится обучение модели, а затем установление неизвестных атрибутов с помощью полученной модели и оценка её качества [3]. 

В данной работе рассматривается задача предсказания демографических атрибутов пользователей социальной сети ВКонтакте; для признакового описания объектов вводятся признаки, использующие распределения подграфов в графах. 

\section{Постановка задачи}

Цель исследования - ввести распределения подграфов в качестве нового признака для описания объектов (пользователей) и проверить гипотезу о том, что с их помощью можно предсказывать демографические атрибуты, а также сравнить с другими методами - например такими, как использование распределения атрибутов соседей и распространение меток - и выяснить, как введённый признак влияет на качество предсказания.

Для достижения поставленной цели необходимо решить следующие подзадачи:
\begin{itemize}  
  \item Выбрать пользователей с полным набором известных атрибутов (пол, возраст, семейное положение, образование) для обучающей выборки; 
  \item Для пользователей из обучающей выборки постороить графы социальных связей до второй окрестности. Каждому пользователю ставится в соответствие граф $G(V, E)$, в котором $v \in V$ - это либо сам исследуемый пользователь, либо пользователь из списка его друзей и друзей друзей, а ребро $(u, v) \in E$ - дружественная связь между пользователями $u \textnormal{ и } v$; 
  \item Построить вектора-распределения подграфов в полученных графах;
  \item Построить классификационные модели, где в качестве признакового описания объектов рассматриваются признаки, использующие посчитанные ранее распределения подграфов;
  \item Произвести сравнение качества предсказания других методов решения задачи и данного. 
\end{itemize}


%[1].
%Li  Q.,  Kim  B.  M.  Constructing  user  profiles  for  collaborative  recommender  system 
%//Advanced Web Technologies and Applications. – Springer Berlin Heidelberg, 2004. – С.100-110. 
%[2].
%Bharat  K.,  Lawrence  S.,  Sahami  M.  Generating  user  information  for  use  in  targeted 
%advertising : заяв. пат. 10/750,363 США. – 2003.
%[3] Гомзин  А.Г.,  Кузнецов  С.Д.  Методы  построения  социо-
%демографических  профилей  пользователей  сети  Интернет.  Труды  ИСП  РАН,  том  27, 
%вып. 4, 2015 г.,
%стр. 129-144. DOI: 10.15514/ISPRAS-2015-27(4)-7. 

%%%%%%%%%%%%%%%%%%%%%%%%%%%%
%%%%%Ende des Dokuments%%%%%
%%%%%%%%%%%%%%%%%%%%%%%%%%%%
\newpage
\newpage


%\setlength\bibitemsep{\baselineskip}

%\printbibliography
\newpage
%%Nicht anfassen, so ist der Dokumentenaufbau
\addtocontents{toc}{\protect\setcounter{tocdepth}{0}}
\renewcommand{\appendixtocname}{Anhang}
\addappheadtotoc
\renewcommand{\appendixpagename}{Anhang}
\appendices
\appendixpage
\appendixtitleoff
%%%%%%%%%%%%%%%%%%%%%%%%%%%%%
%%%Ab hier Inhalt einfügen%%%
%%%%%%%%%%%%%%%%%%%%%%%%%%%%%



%%%%%%%%%%%%%%%%%%%%%%%%%%%%
%%%%%Ende des Dokuments%%%%%
%%%%%%%%%%%%%%%%%%%%%%%%%%%%
\MSonehalfspacing
\newpage
\restoreapp


\newpage
%\section*{Eigenständigkeitserklärung}


%Hiermit versichere ich, dass ich die Hausarbeit selbstständig verfasst und keine anderen als die angegebenen Quellen und Hilfsmittel benutzt habe, alle Ausführungen, die anderen Schriften wörtlich oder sinngemäß entnommen wurden, kenntlich gemacht sind und die Arbeit in gleicher oder ähnlicher Fassung noch nicht Bestandteil einer Studien- oder Prüfungsleistung war.

% 3,5cm Abstand nach oben
%\vspace{100mm}
% Ort, Datum
%\noindent{}STADT, den \today
% 8 cm breite Linie für die Unterschrift
\begin{minipage}[t]{8cm}
% gepunktete Linie
%\centering \hspace{20mm} \hrulefill \\
% Text unter der Linie
%\hspace{20mm}VORNAME NACHNAME
\end{minipage}
\end{document}